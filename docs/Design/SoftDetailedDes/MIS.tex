\documentclass[12pt, titlepage]{article}

\usepackage{amsmath, mathtools}

\usepackage[round]{natbib}
\usepackage{amsfonts}
\usepackage{amssymb}
\usepackage{graphicx}
\usepackage{colortbl}
\usepackage{xr}
\usepackage{hyperref}
\usepackage{longtable}
\usepackage{xfrac}
\usepackage{tabularx}
\usepackage{float}
\usepackage{siunitx}
\usepackage{booktabs}
\usepackage{multirow}
\usepackage[section]{placeins}
\usepackage{caption}
\usepackage{fullpage}

\hypersetup{
bookmarks=true,     % show bookmarks bar?
colorlinks=true,       % false: boxed links; true: colored links
linkcolor=red,          % color of internal links (change box color with linkbordercolor)
citecolor=blue,      % color of links to bibliography
filecolor=magenta,  % color of file links
urlcolor=cyan          % color of external links
}

\usepackage{array}

\externaldocument{../../SRS/SRS}

%% Comments

\usepackage{color}

\newif\ifcomments\commentstrue %displays comments
%\newif\ifcomments\commentsfalse %so that comments do not display

\ifcomments
\newcommand{\authornote}[3]{\textcolor{#1}{[#3 ---#2]}}
\newcommand{\todo}[1]{\textcolor{red}{[TODO: #1]}}
\else
\newcommand{\authornote}[3]{}
\newcommand{\todo}[1]{}
\fi

\newcommand{\wss}[1]{\authornote{blue}{SS}{#1}} 
\newcommand{\plt}[1]{\authornote{magenta}{TPLT}{#1}} %For explanation of the template
\newcommand{\an}[1]{\authornote{cyan}{Author}{#1}}

%% Common Parts

\newcommand{\progname}{Helmholtz-Coil-Current-Calculator-CAS741} % PUT YOUR PROGRAM NAME HERE
\newcommand{\authname}{Reyhaneh Norouziani} % AUTHOR NAMES                  

\usepackage{hyperref}
    \hypersetup{colorlinks=true, linkcolor=blue, citecolor=blue, filecolor=blue,
                urlcolor=blue, unicode=false}
    \urlstyle{same}
                                


\begin{document}

\title{Module Interface Specification for \progname{}}

\author{\authname}

\date{\today}

\maketitle

\pagenumbering{roman}

\section{Revision History}

\begin{tabularx}{\textwidth}{p{3cm}p{2cm}X}
\toprule {\bf Date} & {\bf Version} & {\bf Notes}\\
\midrule
Date 1 & 1.0 & Notes\\
Date 2 & 1.1 & Notes\\
\bottomrule
\end{tabularx}

~\newpage

\section{Symbols, Abbreviations and Acronyms}

See SRS Documentation at \url{https://github.com/rnorouziani/3D-H3C/blob/main/docs/SRS/SRS.pdf}

\newpage

\tableofcontents

\newpage

\pagenumbering{arabic}

\section{Introduction}

The following document details the Module Interface Specifications for \progname{}

Complementary documents include the System Requirement Specifications
and Module Guide.  The full documentation and implementation can be
found at \url{https://github.com/rnorouziani/3D-H3C}

\section{Notation}

The structure of the MIS for modules comes from \citet{HoffmanAndStrooper1995},
with the addition that template modules have been adapted from
\cite{GhezziEtAl2003}.  The mathematical notation comes from Chapter 3 of
\citet{HoffmanAndStrooper1995}.  For instance, the symbol := is used for a
multiple assignment statement and conditional rules follow the form $(c_1
\Rightarrow r_1 | c_2 \Rightarrow r_2 | ... | c_n \Rightarrow r_n )$.

The following table summarizes the primitive data types used by \progname. 

\begin{center}
\renewcommand{\arraystretch}{1.2}
\noindent 
\begin{tabular}{l l p{7.5cm}} 
\toprule 
\textbf{Data Type} & \textbf{Notation} & \textbf{Description}\\ 
\midrule
character & char & a single symbol or digit\\
boolen & bool & true or false \\
natural number & $\mathbb{N}$ & a number without a fractional component in [1, $\infty$) \\
real & $\mathbb{R}$ & any number in (-$\infty$, $\infty$)\\
Abstract Data Type & CoilT & stores coil pair’s configuration data\\
\bottomrule
\end{tabular} 
\end{center}

\noindent
The specification of \progname \ uses some derived data types: sequences, strings, and
tuples. Sequences are lists filled with elements of the same data type. Strings
are sequences of characters. Tuples contain a list of values, potentially of
different types. In addition, \progname \ uses functions, which
are defined by the data types of their inputs and outputs. Local functions are
described by giving their type signature followed by their specification.

\section{Module Decomposition}

The following table is taken directly from the Module Guide document for this project.

\begin{table}[h!]
\centering
\begin{tabular}{p{0.3\textwidth} p{0.6\textwidth}}
\toprule
\textbf{Level 1} & \textbf{Level 2}\\
\midrule

{Hardware-Hiding} & ~ \\
\midrule

\multirow{5}{0.3\textwidth}{Behaviour-Hiding} & Input Parameters\\
& Output Format Module\\
& Output Verification Module\\
& Magnetic Core Module\\
& CoilT Module\\
& Control Module\\
\midrule

\multirow{1}{0.3\textwidth}{Software Decision} & Vector Module\\

\bottomrule

\end{tabular}
\caption{Module Hierarchy}
\label{TblMH}
\end{table}

\newpage
~\newpage

\section{MIS of Input Format Module} \label{MIF}

\subsection{Module}
Input

\subsection{Uses}
Hardware Hiding Modules

\subsection{Syntax}

\subsubsection{Exported Constants}
None.
\subsubsection{Exported Access Programs}

\begin{center}
\begin{tabular}{p{3cm} p{3cm} p{4cm} p{2cm}}
\hline
\textbf{Name} & \textbf{In} & \textbf{Out} & \textbf{Exceptions} \\
\hline
load\_coils & string & [CoilT, CoilT, CoilT, [$\mathbb{R}$, $\mathbb{R}$, $\mathbb{R}$]] & FileError  \\
\hline
load\_params & string & [[$\mathbb{R}$, $\mathbb{R}$, $\mathbb{R}$], [$\mathbb{R}$, $\mathbb{R}$, $\mathbb{R}$]]  & FileError  \\
\hline
\end{tabular}
\end{center}

\subsection{Semantics}

\subsubsection{State Variables}
None.
\subsubsection{Environment Variables}
command line:string, file:string
\subsubsection{Assumptions}
None.

\subsubsection{Access Routine Semantics}
\noindent load\_coils(file\_name):
\begin{itemize}
\item output: The output is composed of an array containing three instances of CoilT, with each instance representing a pair of coils, and an array representing the magnetic moment of the obeject subject to the specified target magnetic torque and force .  
\item exception: FileError if the file name file\_name cannot be found OR the format of inputFile is incorrect.
\end{itemize}
\noindent load\_params(input\_str):
\begin{itemize}
\item output: The output is an array composed of tow arrays. One representing the target force vector, and the other representing the target torque vector.
\item exception: FileError if the format of input\_str is incorrect.
\end{itemize}

\subsubsection{Local Functions}
None.


\section{MIS of CoilT module} \label{MCT}

\subsection{Module}
coil

\subsection{Uses}
None.

\subsection{Syntax}

\subsubsection{Exported Constants}
None.
\subsubsection{Exported Access Programs}

\begin{center}
\begin{tabular}{p{4cm} p{4cm} p{4cm} p{2cm}}
\hline
\textbf{Name} & \textbf{In} & \textbf{Out} & \textbf{Exceptions} \\
\hline
CoilT & $\mathbb{R}$, $\mathbb{R}$, $\mathbb{N}$, $\mathbb{R}$ & CoilT  & ValueError \\
\hline
get\_R & - & $\mathbb{R}$  & -  \\
\hline
get\_L & - & $\mathbb{R}$  & -  \\
\hline
get\_N & - & $\mathbb{N}$  & -  \\
\hline
get\_maxI & - & $\mathbb{R}$  & -  \\
\hline
get\_magnetic\_force & $\mathbb{R}$, [$\mathbb{R}$, $\mathbb{R}$, $\mathbb{R}$] & [$\mathbb{R}$, $\mathbb{R}$, $\mathbb{R}$]  & -  \\
\hline
get\_magnetic\_torque & $\mathbb{R}$, [$\mathbb{R}$, $\mathbb{R}$, $\mathbb{R}$] & [$\mathbb{R}$, $\mathbb{R}$, $\mathbb{R}$]  & -  \\
\hline
\end{tabular}
\end{center}

\subsection{Semantics}

\subsubsection{State Variables}
R, l, N, Max\_I
\subsubsection{Environment Variables}
None.
\subsubsection{Assumptions}
None.

\subsubsection{Access Routine Semantics}

\noindent CoilT(R,l,N,Max\_I):
\begin{itemize}
\item transition: The state variables are initialized with their values by invoking the constructor.
\item exception: ValueError if the inputs do not adhere to the Input Data Constraints. Detailed messages corresponding to each type of violation can be found in the appendix.
\end{itemize}

\noindent get\_R():
\begin{itemize}
\item output: The radius of the coil is returned as the output.
\end{itemize}

\noindent get\_R():
\begin{itemize}
\item output: The radius of the coil is returned as the output.
\end{itemize}
\noindent get\_L():
\begin{itemize}
\item output: The distance between the pair of coils is returned as the output.
\end{itemize}
\noindent get\_N():
\begin{itemize}
\item output: The number of turns of wire in each coil is returned as the output.
\end{itemize}
\noindent get\_maxI():
\begin{itemize}
\item output:  The maximum acceptable electric current of the coil is returned as the output.
\end{itemize}
\noindent get\_magnetic\_fild(I, isMaxwell):
\begin{itemize}
\item output: Calculate the magnetic field generated in the coils by the current, I. The parameter isMaxwell determines whether the coil configuration is that of a Helmholtz coil or a Maxwell coil. The equation mentioned in GD2 and GD3 of the \href{https://github.com/rnorouziani/3D-H3C/blob/main/docs/SRS/SRS.pdf}{System Requirements Specification (SRS)}  is used. 
\end{itemize}
\noindent get\_magnetic\_force(I:$\mathbb{R}$, [$m_x$:$\mathbb{R}$, $m_y$:$\mathbb{R}$, $m_z$:$\mathbb{R}$]):
\begin{itemize}
\item output: The magnetic force exerted on an object, characterized by its magnetic moment vector m, is calculated based on the magnetic field produced by the current I.
\end{itemize}
\noindent get\_magnetic\_torque(I:$\mathbb{R}$, [$m_x$:$\mathbb{R}$, $m_y$:$\mathbb{R}$, $m_z$:$\mathbb{R}$]):
\begin{itemize}
\item output: The magnetic torque exerted on an object, characterized by its magnetic moment vector m, is calculated based on the magnetic field produced by the current I.
\end{itemize}

\subsubsection{Local Functions}
None.



\section{MIS of Magnetic Core Module} \label{MMC}

\subsection{Module}
mCore

\subsection{Uses}
Vector Module, CoilT Module [\ref{MCT}]

\subsection{Syntax}

\subsubsection{Exported Constants}
None.
\subsubsection{Exported Access Programs}

\begin{center}
\begin{tabular}{p{6cm} p{4cm} p{3cm} p{2cm}}
\hline
\textbf{Name} & \textbf{In} & \textbf{Out} & \textbf{Exceptions} \\
\hline
MagneticCore  & CoilT, CoilT, CoilT & MagneticCore   & - \\
\hline
calculate\_current\_of\_target\_force & - & $\mathbb{N}$  & -  \\
\hline
calculate\_current\_of\_target\_torque & [$\mathbb{R}$, $\mathbb{R}$, $\mathbb{R}$] & $\mathbb{R}$  & -  \\
\hline
\end{tabular}
\end{center}

\subsection{Semantics}

\subsubsection{State Variables}
CoilTx:CoilT,  CoilTy:CoilT,  CoilTz:CoilT , [$m_x$:$\mathbb{R}$, $m_y$:$\mathbb{R}$, $m_z$:$\mathbb{R}$]
\subsubsection{Environment Variables}
None.
\subsubsection{Assumptions}
None.

\subsubsection{Access Routine Semantics}

\noindent MagneticCore (CoilT, CoilT, CoilT, [$\mathbb{R}$, $\mathbb{R}$, $\mathbb{R}$]):
\begin{itemize}
\item transition: The state variables are initialized with their values by invoking the constructor.
\end{itemize}

\noindent calculate\_current\_of\_target\_force([$\mathbb{R}$, $\mathbb{R}$, $\mathbb{R}$]):
\begin{itemize}    
\item output: The current is calculated by using the equation mentioned in IM1 in the \href{https://github.com/rnorouziani/3D-H3C/blob/main/docs/SRS/SRS.pdf}{System Requirements Specification (SRS)} document. The resulting calculated current is then returned as the output.
\end{itemize}

\noindent calculate\_current\_of\_target\_torque([$\mathbb{R}$, $\mathbb{R}$, $\mathbb{R}$]):
\begin{itemize}
\item output: The current is calculated by using the equation mentioned in IM2 in the \href{https://github.com/rnorouziani/3D-H3C/blob/main/docs/SRS/SRS.pdf}{System Requirements Specification (SRS)} document. The resulting calculated current is then returned as the output.
\end{itemize}

\subsubsection{Local Functions}
None.
\section{MIS of Output Verification Module} \label{MOV}

\subsection{Module}
Input

\subsection{Uses}
Hardware Hiding Modules

\subsection{Syntax}

\subsubsection{Exported Constants}
None.
\subsubsection{Exported Access Programs}

\begin{center}
\begin{tabular}{p{4cm} p{3cm} p{4cm} p{2cm}}
\hline
\textbf{Name} & \textbf{In} & \textbf{Out} & \textbf{Exceptions} \\
\hline
OutputVerification & CoilT,CoilT,CoilT, [$\mathbb{R}$, $\mathbb{R}$, $\mathbb{R}$], [$\mathbb{R}$, $\mathbb{R}$, $\mathbb{R}$], [$\mathbb{R}$, $\mathbb{R}$, $\mathbb{R}$], $\mathbb{R}$, $\mathbb{R}$ & - & OutputVerification  \\
\hline
is\_currents\_within\_range & - & bool & -  \\
\hline

calculate\_accuracy & - & bool & -  \\
\hline
\end{tabular}
\end{center}

\subsection{Semantics}

\subsubsection{State Variables}
CoilTx:CoilT,  CoilTy:CoilT,  CoilTz:CoilT , [$m_x$:$\mathbb{R}$, $m_y$:$\mathbb{R}$, $m_z$:$\mathbb{R}$], [$f_x$:$\mathbb{R}$, $f_y$:$\mathbb{R}$, $f_z$:$\mathbb{R}$], [$t_x$:$\mathbb{R}$, $t_y$:$\mathbb{R}$, $t_z$:$\mathbb{R}$], I1:$\mathbb{R}$, I2:$\mathbb{R}$
\subsubsection{Environment Variables}

\subsubsection{Assumptions}
None.

\subsubsection{Access Routine Semantics}
\noindent OutputVerification(CoilTx,  CoilTy,  CoilTz , [$m_x$:$\mathbb{R}$, $m_y$:$\mathbb{R}$, $m_z$:$\mathbb{R}$], [$f_x$:$\mathbb{R}$, $f_y$:$\mathbb{R}$, $f_z$:$\mathbb{R}$], [$t_x$:$\mathbb{R}$, $t_y$:$\mathbb{R}$, $t_z$:$\mathbb{R}$], I1:$\mathbb{R}$, I2:$\mathbb{R}$):
\begin{itemize}
\item output: The state variables are initialized with their values by invoking the constructor.
\end{itemize}

\noindent is\_currents\_within\_range():
\begin{itemize}
\item output: Returns true if all the currents are smaller than the maximum acceptable electric current for the corresponding coils.
\end{itemize}

\noindent calculate\_accuracy():
\begin{itemize}
\item output: Returns true if the accuracy of the calculated current I exceeds a specified constant value.
\end{itemize}
\subsubsection{Local Functions}
None.


\section{MIS of Control Module} \label{Mc}

\subsection{Module}
control

\subsection{Uses}
Input Format Module [\ref{MIF}], Output Format Module, Output Verification Module [\ref{MOV}], Magnetic Core Module [\ref{MMC}], CoilT Module [\ref{MCT}]

\subsection{Syntax}

\subsubsection{Exported Constants}
None.
\subsubsection{Exported Access Programs}

\begin{center}
\begin{tabular}{p{6cm} p{4cm} p{3cm} p{2cm}}
\hline
\textbf{Name} & \textbf{In} & \textbf{Out} & \textbf{Exceptions} \\
\hline
main  & - & - & - \\
\hline
\end{tabular}
\end{center}

\subsection{Semantics}

\subsubsection{State Variables}
CoilTx:CoilT,  CoilTy:CoilT,  CoilTz:CoilT , inputF:InputFormat, magneticCore:MagneticCore, verification:OutputVerification, OutputF:OutputFormat
\subsubsection{Environment Variables}
None.
\subsubsection{Assumptions}
None.

\subsubsection{Access Routine Semantics}

\noindent main():
\begin{itemize}
\item transition: The state variables are initialized with their values through the invocation of constructors, and these state variables are modified during the execution by these steps:\\

\text{[coilTx:CoilT, coilTy:CoilT, coilTz:CoilT,[$m_x$:$\mathbb{R}$, $m_y$:$\mathbb{R}$, $m_z$:$\mathbb{R}$]] = load\_coils(file\_name:string)}\\
\\
The following steps are repeatable, enabling the calculation of currents multiple times.\\
\\
\text{[[$f_x$:$\mathbb{R}$, $f_y$:$\mathbb{R}$, $f_z$:$\mathbb{R}$], [$t_x$:$\mathbb{R}$, $t_y$:$\mathbb{R}$, $t_z$:$\mathbb{R}$]] = load\_params(file\_name:string)}\\

MagneticCore(coilTx, coilTy, coilTz)\\
\\
I1 = calculate\_current\_of\_target\_force()\\
\\
I2 = calculate\_current of\_target\_torque()\\
\\
OutputVerification(CoilTx:CoilT,  CoilTy:CoilT,  CoilTz:CoilT, [$m_x$:$\mathbb{R}$, $m_y$:$\mathbb{R}$, $m_z$:$\mathbb{R}$], [$f_x$:$\mathbb{R}$, $f_y$:$\mathbb{R}$, $f_z$:$\mathbb{R}$], [$t_x$:$\mathbb{R}$, $t_y$:$\mathbb{R}$, $t_z$:$\mathbb{R}$], I1:$\mathbb{R}$, I2:$\mathbb{R}$)\\
\\
is\_currents\_within\_range()\\
\\
calculate\_accuracy()\\

\end{itemize}

\subsubsection{Local Functions}
None.
\newpage

\bibliographystyle {plainnat}
\bibliography {../../../refs/References}

\newpage

\section{Appendix} \label{Appendix}


\begin{longtable}{l p{12cm}}
\caption{Possible Exceptions} \\
\toprule
\textbf{Exception Type} & \textbf{Error Message} \\
\midrule
ValueError & Error: Radius must be $> 0$ \\
ValueError & Error: Distance between the coils must be $> 0$ \\
ValueError & Error: The number of turns must be elements of the set of natural numbers\\
\bottomrule
\end{longtable}


\end{document}
